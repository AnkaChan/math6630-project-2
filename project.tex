\documentclass[11pt]{amsart}

\usepackage{amsmath,amssymb,graphicx,bbm}
\usepackage{amsthm,verbatim}
\usepackage{mathrsfs,mathtools}

\usepackage[footnotesize,bf]{caption}
\usepackage[left=1.1in,right=1.1in,top=1in]{geometry}

\newcommand{\bs}[1]{\boldsymbol{#1}}
\newcommand{\mathd}{\textrm{d}}
\newcommand{\ddx}[1]{\frac{\mathd}{\mathd #1}}
\newcommand{\N}{\mathbbm{N}}
\newcommand{\R}{\mathbbm{R}}
\newcommand{\pfpx}[2]{\frac{\partial #1}{\partial #2}}

%% Patch for amsart date
\usepackage{etoolbox}
\makeatletter
\patchcmd{\@maketitle}
  {\ifx\@empty\@dedicatory}
  {\ifx\@empty\@date \else {\vskip3ex \centering\footnotesize\@date\par\vskip1ex}\fi
   \ifx\@empty\@dedicatory}
  {}{}
\patchcmd{\@adminfootnotes}
  {\ifx\@empty\@date\else \@footnotetext{\@setdate}\fi}
  {}{}{}
\makeatother
%%

\title{Project 2: Spectral and finite volume methods}
\author{XXXXXXX}
\date{\today}

\begin{document}
\maketitle

\textbf{1.} (Fourier spectral methods)\\
Consider the partial differential equation,
\begin{align}\label{eq:advdiff}
  u_t + \sin (\pi x) u_x &= \frac{1}{2} u_{x x}, & u(x,0) &= \sin (2 \pi x), %& u(0,t) = u(1, t) &= 0,
\end{align}
for $u = u(x,t)$, $x \in [0,1]$, $t > 0$, and periodic boundary conditions.
  \begin{itemize}
     \item[(a)] Write the semi-discrete form for the Fourier-Galerkin scheme for this problem. You may use a complex-valued basis (e.g., $e^{i j x},\; i = \sqrt{-1}$) if you like, but the solution $u$ should be real-valued. Use an odd number $2 N + 1$ for the dimension of the spatial variable discretization space.
    \item[(b)] If using Forward Euler for fully discretizing the semi-discrete form, what is the timestep restriction (say as a function of $N$) to ensure that the scheme is stable in the sense of $A$-stability? (You may investigate this analytically or empirically via the region of stability.) What aspect of the equation \eqref{eq:advdiff} is the dominant contributor to the timestep restriction?  How does this make the total computational time of the solver scale with $N$?
    \item[(c)] Implement both implicit (e.g., Crank-Nicolson) and explicit time-stepping methods (say RK-2) for solving the above problem up to time $T = 1$. Visualize the simulated results and discuss the advantages and disadvantages of each approach.
  \end{itemize}

  \textbf{2.} (Strang splitting)\\
  Consider the Korteweg de Vries equation:
  \begin{align}\label{eq:kdv}
    u_t + u_{xxx} - 6 u u_x&= 0, & u(x,0) &= -2 \mathrm{sech}^2(x),
  \end{align}
  over the domain $x \in [-10, 10]$ up to terminal time $T=1$, with periodic boundary conditions.
  \begin{itemize}
     \item[(a)] Consider an appropriate Fourier collocation discretization along with an explicit time-stepping scheme for solving \eqref{eq:kdv}. What are the advantages and disadvantages of this type of approach? (You need not actually implement this method.)
     \item[(b)] With the same context as part (a), consider instead an impicit time-stepping scheme for solving \eqref{eq:kdv}. What are the advantages and disadvantages of this type of approach? (Again, no implementation is necessary.)
    \item[(c)] Consider a general semi-discrete ODE,
      \begin{align}\label{eq:split-ode}
        \bs{u}' &= \mathcal{N}(\bs{u}) + \mathcal{L}(\bs{u}),
      \end{align}
      where $\mathcal{L}$ and $\mathcal{N}$ are some operators. \textit{Strang splitting} is an alternating approach for numerically solving the above ODE that treats $\mathcal{L}$ and $\mathcal{N}$ with different time discretizations and evolves them independently. To communicate the main idea, consider the following two schemes for solving two simpler ODE's:
      \begin{align*}
        \bs{f}' &= \mathcal{N}(\bs{f}) \hskip 5pt \xrightarrow{\textrm{RK2/Midpoint method (explicit)}}  \hskip 5pt\bs{f}_{n+1} = \mathcal{S}_{1,k} (\bs{f}_n) \coloneqq \bs{f}_n + k \mathcal{N}\left(\bs{f}_n + \frac{k}{2} \mathcal{N}(\bs{f}_n)\right) \\
        \bs{g}' &= \mathcal{L}(\bs{g}) \hskip 5pt \xrightarrow{\textrm{Crank-Nicolson (implicit)}} \hskip 5pt \bs{g}_{n+1} = \mathcal{S}_{2,k} (\bs{g}_n),
      \end{align*}
      where the operator $\mathcal{S}_{2,k}$ defines the solution to the implicit Crank-Nicolson iteration:
      \begin{align*}
        \bs{g}_{n+1} = \mathcal{S}_{2,k}(\bs{g}_n) \hskip 5pt \Longrightarrow\hskip 5pt \bs{g}_{n+1} = \bs{g}_n + \frac{k}{2} \mathcal{L}\left(\bs{g}_n\right) + \frac{k}{2} \mathcal{L}(\bs{g}_{n+1}).
      \end{align*}
      With these individual schemes, Strang splitting is the following time-stepping for \eqref{eq:split-ode}:
      \begin{align*}
        \bs{w}_1 &= \mathcal{S}_{2,k/2} (\bs{u}_n) \\
        \bs{w}_2 &= \mathcal{S}_{1,k} (\bs{w}_1) \\
        \bs{u}_{n+1} &= \mathcal{S}_{2,k/2} (\bs{w}_2)
      \end{align*}
      Given this idea, how would this help in alleviating some of the disadvantages in parts (a) and (b) with purely explicit or purely implicit methods for solving \eqref{eq:kdv}? Propose a scheme using Strang splitting (say using the $\mathcal{L}_k$ and $\mathcal{N}_k$ operators above) for solving \eqref{eq:kdv}.
    \item[(d)] Implement your scheme from (c) and illustrate how the disadvantages from (a) and (b) are ameliorated with this approach.
  \end{itemize}


\textbf{3.} (Legendre spectral methods)\\
Consider the viscous Burgers' partial differential equation,
  {\begin{align*}
      u_t + \left(\frac{u^2}{2}\right)_x &= \nu u_{x x}, \hskip 5pt x \in (-1,1)\\
      u(\pm 1, t) &= 0. \\
      u(x,0) &= \sin (\pi x).
  \end{align*}}
  Implement both a Legendre-Galerkin and Legendre-collocation solver for this equation. You may use an explicit time-stepping method.
  \begin{itemize}
    \item For \textit{viscous} Burgers', $\nu > 0$, show and discuss results for $t > 0$ when $\nu$ is small, and when $\nu$ is large.
    \item For \textit{inviscid} Burgers', $\nu = 0$, what differences do you observe between the collocation and Galerkin methods? Do the solutions appear to be accurate for large $t$?
  \end{itemize}

\textbf{4.} (Biharmonic equation)\\
Consider the biharmonic differential equation
  {\begin{align*}
    u^{(4)} - u_{x x} + u &= f(x), & x &\in (-1,1) \\
    u(-1) = u(1) &= 0, & u'(-1) = u'(1) &= 0.
  \end{align*}}
  Consider a Galerkin procedure to find a solution $u$ from the polynomial space
  \begin{align*}
    P_{N,0,0} = \left\{ p\; \big|\; \deg p \leq N, \;\; p(-1) = p(1) = p'(-1) = p'(1) = 0\right\},
  \end{align*}
  which is a subspace of $H_0^2([-1,1])$. Derive an appropriate bilinear form for the differential equation using this space. Numerically implement a Galerkin solver for this equation, using basis functions $v_n$ for $P_{N,0,0}$ defined as
  {\begin{align*}
      v_n(x) &= \sum_{j=n-1}^{n+3} c_{n,j} p_{j}(x), & n &\geq 1.
  \end{align*}}
  where $c_{n,j}$ are chosen so that $v_n \in P_{N,0,0}$. Demonstrate exponential accuracy of your solver as a function of $N$ (using, e.g., the method of manufactured solutions).\\[12pt]


\textbf{5.} (Multidimensional wave equation)\\
Consider the partial differential equation
  {\begin{align*}
      u_t = \boldsymbol{c} \cdot \nabla u(x,y),
    \end{align*}
  }
  with periodic boundary conditions on $(x,y) \in [0, 2\pi)^2$. Let the wavespeed be
    {
    \begin{align*}
      \boldsymbol{c}(x,y) = \left( \sin (x + y) , \sin (x - y) \right).
    \end{align*}
  }
  \begin{itemize}
    \item Code a Fourier-Galerkin and a Fourier-collocation method for this PDE.
    \item Discuss the computational complexity of each of your solvers.
    \item Investigate the accuracy of your solver, both in terms of timestep size and number of polynomial terms $N$. (E.g., by using an extremely refined comptuational solution as the ``exact" solution.)
  \end{itemize}
  \vskip 10pt

  \textbf{6.} (1D Euler equations)\\
  This problem asks you to write a solver for the one-dimensional Euler equations of gas dynamics. These equations, in conservation form, are written as
\begin{align*}
  \pfpx{\bs{u}}{t} + \pfpx{\bs{f(u)}}{x} &= 0, & t > 0,
\end{align*}
where the conserved variable $\bs{u}(x,t) \in \R^3$ and flux $\bs{f}$ are functions of the physical gas density $\rho(x,t)$, velocity $u(x,t)$, pressure $p(x,t)$, and energy $E(x,t)$,
\begin{align*}
\bs{u} &= \left[ \begin{array}{c} \rho \\ \rho u \\ E \end{array}\right], & \bs{f} &= \left[\begin{array}{c} \rho u \\ \rho u^2 + p \\ (E + p) u \end{array}\right],
\end{align*}
where pressure and energy are related by
\begin{align*}
  p = \left(\gamma - 1\right) \left( E - \frac{1}{2} \rho u^2 \right),
\end{align*}
where $\gamma = 7/5$ is a physical constant.
The flux Jacobian for this problem is given by
\begin{align*}
  \pfpx{\bs{f}}{\bs{u}} = \left[ \begin{array}{ccc}
                                0 & 1 & 0 \\
      -\frac{3 - \gamma}{2} u^2 & (3 - \gamma) u & \gamma - 1 \\
  -\frac{\gamma E u}{\rho} + (\gamma - 1) u^3 & \frac{\gamma E}{\rho} - \frac{3(\gamma-1) u^2}{2} & \gamma u \end{array}\right]
\end{align*}
This Jacobian can be diagonalized as follows:
\begin{align*}
  \pfpx{\bs{f}}{\bs{u}} &= \bs{S}(\bs{u}) \bs{\Lambda}(\bs{u}) \bs{S}(\bs{u})^{-1}, \\
  \bs{\Lambda} &= \left[\begin{array}{ccc}
                   u+c & 0 & 0 \\
                   0 & u & 0 \\
                   0 & 0 & u-c \end{array}\right].
\end{align*}
We have introduced the local sound speed $c$, defined as 
\begin{align*}
  c &\coloneqq \sqrt{\frac{\gamma p}{\rho}}, 
\end{align*}

Implement a finite volume solver for this problem using a Harten-Lax-van Leer (HLL) flux. This flux, given left- and right-hand states $\bs{u}_l$ and $\bs{u}_r$, respectively, is evaluated as
\begin{align*}
  \bs{F}(\bs{u}_l,\bs{u}_r) = \left\{ \begin{array}{rl} \bs{f}(\bs{u}_l), & s_- \geq 0 \\
                                                        \frac{s_+ \bs{f}(\bs{u}_l) - s_- \bs{f}(\bs{u}_r) + s_+ s_- (\bs{u}_r - \bs{u}_l)}{s_+ - s_-}, & s_- \leq 0 \leq s_+ \\
\bs{f}(\bs{u}_r), & s_+ \leq 0.\end{array}\right.
\end{align*}
Here, $s_{\pm}$, are the extreme wavespeeds of flux jacobians:
\begin{align*}
  s_- &\coloneqq \min_{i} \min \left\{ \lambda_i\left(\pfpx{\bs{f}}{\bs{u}}\left(\bs{u}_l\right)\right), \lambda_i\left(\pfpx{\bs{f}}{\bs{u}}\left(\bs{u}_r\right) \right)\right\}, &
  s_+ &\coloneqq \max_{i} \max \left\{ \lambda_i\left(\pfpx{\bs{f}}{\bs{u}}\left(\bs{u}_l\right)\right), \lambda_i\left(\pfpx{\bs{f}}{\bs{u}}\left(\bs{u}_r\right) \right)\right\},
\end{align*}
where $\lambda_i(\bs{A})$ denotes the $i$th eigenvalue of $\bs{A}$.

Test your solver with two popular initial conditions:
\begin{itemize}
  \item[(a)] \textit{Sod's shock tube} --- For $x \in [0,1]$,
    {\begin{align*}
      \rho(x,0) &= \left\{ \begin{array}{rl} 1, & x < \frac{1}{2} \\ \frac{1}{8}, & x \geq \frac{1}{2} \end{array}\right. &
        \rho u(x,0) &= 0, &
      E(x,0) &= \frac{1}{\gamma-1}\left\{ \begin{array}{rl} 1, & x < \frac{1}{2} \\ \frac{1}{10}, & x \geq \frac{1}{2} \end{array}\right.
    \end{align*}}
    Use non-periodic boundary conditions that match the $x = 0$ and $x=1$ values of the initial data, and solve up to terminal time $T = 0.2$. Plot the density $\rho$, the velocity $u$, and the pressure $p$.
  \item[(b)] \textit{Shock entropy wave} --- For $x \in [-5, 5]$,
    {\begin{align*}
        \rho &= \left\{ \begin{array}{rl} 3.857143, & x < -4, \\ 1 + 0.2 \sin(\pi x), & x \geq -4 \end{array}\right. \\
        u &= \left\{ \begin{array}{rl} 2.629369, & x < -4, \\ 0, & x \geq -4 \end{array}\right. \\
        p &= \left\{ \begin{array}{rl} 10.33333, & x < -4, \\ 1, & x \geq -4 \end{array}\right.
    \end{align*}}
    Again use non-periodic boundary conditions that match the $x = \pm 5$ values of the initial data. Simulate up to time $T = 1.8$ and again plot the density, velocity, and pressure.
\end{itemize}


\end{document}
